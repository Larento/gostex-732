\documentclass{gostdoc}
\begin{document}

\titlepage

\toc

\abstract
\totalpages, \totalfigures, \totaltables, \totalappendices
\begin{itemlist}
    \item Первый
    \item Второй элемента списка
    \item Еще один
    \item Элемент списка первого уровня, который не умещается на одной строке и, соответственно, распологается на нескольких строках. Начнем вложенный список:
    \begin{itemlist}
        \item Элемент списка второго уровня. Начнем вложенный список третьего уровня:
        \begin{itemlist}
            \item Первый элемент списка третьего уровня
            \item Второй элемент списка третьего уровня
            \item Последний элемент списка третьего и последнего уровня вложенности списков. Этот элемент также располагается на нескольких строках
        \end{itemlist}
        \item Это опять элемент, располагающийся на нескольких строках, однако уже списка второго уровня
        \item Еще один элемент. Начнем еще один вложенный список третьего уровня:
        \begin{itemlist}
            \item Первый элемент списка третьего уровня
            \item Второй элемент списка третьего уровня
            \item Последний элемент списка третьего и последнего уровня вложенности списков. Этот элемент также располагается на нескольких строках
        \end{itemlist}
    \end{itemlist}
    \item Элемент списка первого уровня
\end{itemlist}

\task
\lipsum[12]

\introduction
Основная цель практики – изучить устроийство профилегибочного стана, расположенного в лаборатории имени А.И. Целикова кафедры МТ10 и составить графические материалы в виде плакатов и видеороликов, поясняющих процесс сборки узлов стана и его настройки.

Задачами для достижения цели являются ознакомление с доступной конструкторской документации на профилегибочный стан и изучение способов регулировки рабочего инструмента стана (роликов) в виде, осуществленном в реальном стане.

\section{Заголовок}
\lipsum[1-2]

\subsection{Подзаголовок}
\lipsum[5-6]

\subsubsection{Подподзаголовок}
\lipsum[10]

\section{Новый раздел}
\lipsum[4-5]

% \clearpage
\subsection{Подраздел}
Попробуем графон (Рисунок \ref{fig:themaze71}):

\insertfigure[0.7\linewidth]{themaze71}{Надпись под рисунком надпись под рисунком надпись под рисунком надпись под рисунком надпись под рисунком}

\lipsum[2-4]

Теперь проверим как работают таблицы (таблица \ref{tab:info}):

\begin{inserttable}[1.15]{info}{Надпись над таблицей надпись над таблицей надпись над таблицей надпись над таблицей надпись над таблицей}
    \begin{tabular}{|l|c|}
        \hline
        afsfgdfcbcbc       & 2 \\
        \hline
        fbvgggnv nvgfgfgfg & 4 \\
        \hline
    \end{tabular}
\end{inserttable}

\lipsum[1]

\subsection{Test}
\lipsum[1]

\insertfigurewithdesc{premaze}{This is really cool!}{Надпись под рисунком надпись под рисунком надпись под рисунком надпись под рисунком надпись под рисунком}

\conclusion
Вернемся к рисунку \ref{fig:premaze}.

\lipsum[11]

\appendix
\lipsum[16]

"Hey Lois?" -- "What?!" (рисунок \ref{fig:peter})

\insertfigurewithdesc{peter}{This is Beter Griffon. He is a nice guy.}{Peter Grin in his natural habitat}

\begin{numberedlist}
    \item Hey
    \item now
    \item you're:
    \begin{numberedlist}
        \item a
        \item rockstar,
        \item get
        \item your
        \item game
        \item on
        \begin{numberedlist}
            \item go
            \item play
        \end{numberedlist}
    \end{numberedlist}
    \item hey
    \item now
    \item you're
    \item an
    \item allstar
    \begin{numberedlist}
        \item go
        \item get
        \item paid
        \item get
        \item laid
        \begin{numberedlist}
            \item gatorade
            \item am i right fellas?
            \item up top
        \end{numberedlist}
    \end{numberedlist}
\end{numberedlist}

\appendix
\lipsum[4-6]

\appendix
Hey man, this is таблица \ref{tab:grieferjesus}

\begin{inserttable}[1.15]{grieferjesus}{Run's dead}
    \begin{tabular}{|l|c|}
        \hline
        Hewwo everywone & 2             \\
        \hline
        This is         & 4             \\
        \hline
        Runnin on empty & 69            \\
        \hline
        Food review     & *clap* *clap* \\
        \hline
    \end{tabular}
\end{inserttable}

\lipsum[3-5]

Примеры математических формул, выражений и определений. Обычное выражение в тексте: \( a=5 \)

$\displaystyle \SI[]{0.1}{\per\second\per\meter}$

\end{document}